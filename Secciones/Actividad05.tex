\section{CONCLUSIONES} 
\begin{flushleft}
\textbf{}\\
• BI, proporciona una manera rápida y efectiva de recopilar, abstraer, presentar, formatear y distribuir la información de sus fuentes de datos corporativos, permitiendo a los profesionales de la empresa, tanto dentro como fuera de la organización, visualizar y analizar datos precisos sobre las actividades fundamentales del negocio y utilizarlos para mejorar la toma de decisiones y la planificación estratégica. (Zúmel 2008) \textbf{}\\
\textbf{}\\
• Una nueva forma de implementar BI dentro de las organizaciones, es la utilización de BI Governance, que combina las técnicas de BI, con el manifiesto Ágil, IT Governance y Data Governance, y este conjunto de teorías da como resultado el proceso de administración y seguimiento a la implantación de un proyecto de BI. \textbf{}\\
\textbf{}\\
• En esta investigación se conocieron nuevas y diferentes formas de complementar el trabajo con Data Warehouse, optimizando los tiempos de transferencia de los datos del sistema transaccional a la bodega de datos, acompañar el proceso de montaje de un Data Warehouse y un estudio del retorno de la inversión. \textbf{}\\
\textbf{}\\
• La principal enseñanza que se establece con este trabajo es la enorme gama de posibilidades que ofrece BI y sus herramientas, aquí se mostraron casos diferentes en los cuales se puede aplicar BI, en organización de diferentes sectores, con diferentes formas de trabajar, soportadas por sistemas de información particulares a cada una de ellas y con distintos contenidos en sus bases de datos. BI se establece como el siguiente paso a seguir para poner a las empresas en un nivel competitivo.\textbf{}\\


\textbf{}\\
\textbf{}\\
\textbf{}\\
\textbf{}\\
\textbf{}\\

\textbf{}\\
\textbf{}\\
\end{flushleft}