\section{INTRODUCCIÓN} 
\begin{flushleft}
En la actualidad la gran mayoría de las organizaciones cuenta con un sistema de información que soporta gran parte de las actividades diarias propias del sector de negocios en donde se esté desempeñando, este sistema puede ser sencillo o robusto todo depende de las exigencias del negocio, con el transcurso del tiempo estas aplicaciones llegan a tener la historia de la organización, los datos almacenados en las bases de datos, pueden ser utilizados para argumentar la decisión que se quiera tomar. 
Un estudio realizado en Europa por Information Builders Ibéric mostró el costo que tiene la falta de sistemas de toma de decisiones en las organizaciones, según estos datos, el empleado europeo medio pierde una media de 67 minutos diariamente buscando información de la compañía, lo que equivale a un 15,9\% de su jornada laboral. Para una organización de 1.000 empleados que gane unos 50.000 euros al día esto equivale a 7,95 millones de euros al año de salario perdido, todo ello por la búsqueda de información para tomar una decisión. (Zúmel 2008)
El poder competitivo que puede tener una empresa se basa en la calidad y cantidad de la información que sea capaz de usar en la toma de de decisiones; mediante la implementación de Inteligencia de Negocios se proporcionan las herramientas necesarias para aprovechar los datos almacenados en las bases de datos de los sistemas transaccionales para utilizar la información como respaldo a las decisiones, reduciendo el efecto negativo que puede traer consigo una mala determinación. 
La investigación comienza con la definición de BI, sus aplicaciones; adicionalmente se muestran conceptos y trabajos relevantes en algunas de las herramientas para hacer BI, como son Data Warehouse (Bodega de Datos), Olap (Cubos Procesamiento Analítico en Línea), Balance Scorecard (Cuadro de Mando) y Data Mining (Minería de Datos).






\textbf{}\\
\textbf{}\\
\textbf{}\\






 

\end{flushleft}